%!TEX root = ../dokumentation.tex

\addchap{\langabkverz}
%nur verwendete Akronyme werden letztlich im Abkürzungsverzeichnis des Dokuments angezeigt
%Verwendung: 
%		\ac{Abk.}   --> fügt die Abkürzung ein, beim ersten Aufruf wird zusätzlich automatisch die ausgeschriebene Version davor eingefügt bzw. in einer Fußnote (hierfür muss in header.tex \usepackage[printonlyused,footnote]{acronym} stehen) dargestellt
%		\acs{Abk.}   -->  fügt die Abkürzung ein
%		\acf{Abk.}   --> fügt die Abkürzung UND die Erklärung ein
%		\acl{Abk.}   --> fügt nur die Erklärung ein
%		\acp{Abk.}  --> gibt Plural aus (angefügtes 's'); das zusätzliche 'p' funktioniert auch bei obigen Befehlen
%	siehe auch: http://golatex.de/wiki/%5Cacronym
%	
\begin{acronym}[YTMMM]
\setlength{\itemsep}{-\parsep}

\acro{GFS}{Google File System}
\acro{HDFS}{Hadoop Distributed File System}
\acro{MRC}{Map Reduce Class}
\acro{OSDI}{Operating Systems Design and Implementations}

\end{acronym}
