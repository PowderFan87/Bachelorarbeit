%!TEX root = ../dokumentation.tex

%
% vorher in Konsole folgendes aufrufen:
%	makeglossaries makeglossaries dokumentation.acn && makeglossaries dokumentation.glo
%

%
% Glossareintraege --> referenz, name, beschreibung
% Aufruf mit \gls{...}
%
\newglossaryentry{Glossareintrag}{name={Glossareintrag},plural={Glossareinträge},description={Ein Glossar beschreibt verschiedenste Dinge in kurzen Worten}}

\newglossaryentry{Commodity-Hardware}{name={Commodity-Hardware},description={\flqq Computer hardware that is affordable and easy to obtain. Typically it is a low-performance system that is IBM PC-compatible and is capable of running Microsoft Windows, Linux, or MS-DOS without requiring any special devices or equipment.\frqq\footcite{Beal.2015}}}

\newglossaryentry{Git}{name={Git},plural={Git},description={Git ist ein kostenloses System zur Versionskontrolle für kleine wie auch sehr große Projekte ({\url{http://git-scm.com/}})}}

\newglossaryentry{NetBeans}{name={NetBeans},plural={NetBeans},description={The Smarter and Faster Way to Code Quickly and easily develop desktop, mobile and web applications with Java, HTML5, PHP, C/C++ and more. NetBeans IDE is FREE, open source, and has a worldwide community of users and developers ({\url{https://netbeans.org}})}}

\newglossaryentry{Maven}{name={Maven},plural={Maven},description={Apache Maven is a software project management and comprehension tool. Based on the concept of a project object model (POM), Maven can manage a project’s build, reporting and documentation from a central piece of information \\ ({\url{http://maven.apache.org/}})}}