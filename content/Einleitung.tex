%!TEX root = ../dokumentation.tex

\chapter{Einleitung}\label{cha:Einleitung}
Kurze Einleitung. Vorstellung der Firma und Abteilung. Allgemeine Heranführung zum Thema und der Brance (IT Umfeld ist groß. Hier sollte klar werden, dass im weiteren Verlauf der Arbeit das Web Umfeld speziell mit Java als Grundlage dient).

\section{Problemstellung}\label{sec:Problemstellung}
Genaue Beschreibung der Problemstellung. Was fehlt? Warum fehlt es? Welche Auswirkungen hat das Problem auf die tägliche Arbeit?

\section{Motivation}\label{sec:Motivation}
Beschreibung warum dieses Thema gewählt wurde für die Arbeit. Eventuell beschreiben was ich mir selbst erhoffe von der Arbeit und welchen Nährwert ich mir wünsche. \autoref{equ:Formel1} und \ref{equ:Formel2} zeigen...
\begin{align}
	V & = \frac{1}{2} \tau \cdot r^{2} \cdot h
	\label{equ:Formel1} \\
	\frac{12}{43} \cdot V & = x^2 \cdot n
	\label{equ:Formel2}
\end{align}

\section{Zielsetzung}\label{sec:Zielsetzung}
Was ist das Ziel der Arbeit? Was soll das Ergebnis sein?

\section{Abgrenzung}\label{sec:Abgrenzung}
Was soll NICHT Teil der Arbeit sein? Dabei unbedingt beschreiben das die serverseitige Konfiguration von Hadoop und YARN und das Aufsetzung von Clustern ignoriert wird. Dadurch soll klar werden, dass der Fokus auf der Anwendung des Algorithmus auf einen konkreten Fall liegt.