%!TEX root = ../dokumentation.tex

\chapter{Theoretische Grundlagen}\label{cha:Grundlagen}
Kurze Einleitung in das Kapitel. Beschreiben warum die Grundlagen notwendig sind. Auch darüber schreiben, dass nicht ALLE Grundlagen sondern lediglich die für das Verständnis der Arbeit notwendigen erklärt werden.

\section{Überwachung von IT-Infrastrukturen}\label{sec:UeberwachungIT}
Worum geht es bei der Überwachung? Was ist das Ziel? Warum braucht man eine Überwachung? Wie sieht diese i.d.R. Aus? Welche Kernpunkte gibt es in der Überwachung? \\
Hier sollte die Überleitung zum nächsten Kapitel kommen d.h. Logfiles werden als letztes behandelt in diesem Kapitel, damit der Übergang sauber ist.

\section{Bedeutung von Logfiles}\label{sec:BedeutungVonLogfiles}
Was sind Logfiles im allgemeinen? Was ist die Funktion eines Logfiles? Gibt es Standards? Wenn ja welche und wie sehen die aus? Werden die Standards im weiteren Verlauf der Arbeit noch einmal relevant sein (Ja/Nein begründen und erläutern)?

\section{Einführung in MapReduce}\label{sec:EinführungInMapReduce}
Woher kommt MapReduce? Wie funktioniert MapReduce? Stärken/Schwächen aufzeigen. Versuchen den Algorithmus mathematisch zu beschreiben ($O(n)$ Methode, Mengenleere). Hierfür muss noch Literatur gesucht werden. Bisher nur mathematische Beschreibungen im Internet gefunden.

\begin{equation}
	V = \frac{1}{2} \tau \cdot r^{2} \cdot h
\end{equation}

\section{Was ist Hadoop?}\label{sec:WasIstHadoop}
Was gehört alles zu Hadoop? Worauf zieht Hadoop ab? Warum verwende ich Hadoop statt es einfach selbst zu machen? Stärken und Schwächen aufzeigen.

\section{Betrachten von Alternativen}
Alternativen sowohl zu Hadoop als auch zu MapReduce selbst. Was gibt es in diesen Bereichen noch? Wie unterscheiden sich diese?