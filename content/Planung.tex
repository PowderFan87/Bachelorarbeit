%!TEX root = ../dokumentation.tex

\chapter{Planung}\label{cha:Planung}
Bevor mit die Entwicklung der Anwendung begonnen werden kann, müssen die Anforderungen, sowie Ziele und Umgebung festgelegt werden. Die exakte Planung eines ist notwendig, um eine qualitativ hochwertige, lösungsorientierte Anwendung zu entwickeln.

%Beschreiben worum es im Kapitel geht. Warum wird eine Planung gemacht? Was ist das Ziel der Planung?

\section{Anforderungen \& Produktleistungen}
<Schreiben wozu die Ziele gut sein sollen. welche form sollen sie haben. welchen zweck dienen sie usw. Zusätzlich schreiben warum manche aspekte wie die art der logs nicht definiert wurden oder manche ziele eher abstrakt sind>

\subsection{Anforderungen}
Im folgenden werden die Anforderungen beschrieben, welche an die fertige Anwendung gestellt werden. Bei der Definition von Anforderungen oder Zielen ist es hilfreich, sich gedanklich an das Ende des Projektes zu versetzen, und diese Situation dan zu beschreiben. Außerdem sollten alle Anforderungen "`smart"' sein, d.h., sie sind \textbf{s}peziefisch, \textbf{m}essbar, \textbf{a}ttraktiv, \textbf{r}ealistisch und \textbf{t}erminiert.\footcite[Vgl.][S. 48]{Bauer.2014}



%\begin{table}[h]
%	\centering
%	\begin{tabular}{| l | l | l | l | l |}
%		\hline
%		\rowcolor[HTML]{3531FF} 
%		\multicolumn{1}{|l|}{\cellcolor[HTML]{3531FF}{\color[HTML]{FFFFFF} {\bf \#ID}}} & \multicolumn{1}{l|}{\cellcolor[HTML]{3531FF}{\color[HTML]{FFFFFF} {\bf Name}}} & \multicolumn{1}{l|}{\cellcolor[HTML]{3531FF}{\color[HTML]{FFFFFF} {\bf Modul}}} & \multicolumn{1}{l|}{\cellcolor[HTML]{3531FF}{\color[HTML]{FFFFFF} {\bf Beschreibung}}} & \multicolumn{1}{l|}{\cellcolor[HTML]{3531FF}{\color[HTML]{FFFFFF} {\bf Priorität}}} \\ \hline
%		01 & Konfigurierbarkeit & Properties & Die Anwendung muss über eine .properties Datei konfigurierbar sein & 1 \\ \hline
%		02 &                                                                                      &                                                                                      &                                                                                      &                                                                                      \\ \hline
%		03 &                                                                                      &                                                                                      &                                                                                      &                                                                                     \\  \hline
%	\end{tabular}
%	\caption{My caption}
%	\label{my-label}
%\end{table}

\subsection{Produktleistungen}

\section{Definition der Umgebung}
Die Programmierung der Anwendung erfolgt in der Programmiersprache Java, unter verwendung des Hadoop Frameworks für MapReduce in der Version 2.7.1. Die Entwicklung wird mit der \ac{IDE} \gls{NetBeans}, Version 8.0.2, durchgeführt. Für die Versionsverwaltung wird \gls{Git} eingesetzt. In der \ac{IDE} wird ein \gls{Maven}-Projekt erzeugt, um alle benötigten Bibliotheken automatisch, durch die \ac{IDE}, beziehen zu können.

%TODO: Datenbank definieren
Über die \ac{IDE} wird eine ausführtbare JAR-Datei erzeugt, welche auf dem Server mit Hadoop ausgeführt werden soll. Für die Ausführung wird Java in Version 1.7 benötigt. Zusätzlich wird eine <PUTNAM> Datenbank für die Speicheurng der Ausgabe benötigt.

Die anwendung wird durch das Hadoop Framework Version 2.7.1 ausgeführt. Es muss mindestens eine \textit{Single Node} Installation vorhanden sein. Es müssen die entsprechenden Rechte auf dem Dateisystem \ac{HDFS} gesetzt sein, um Dateien lesen und schreiben zu können.

%Welche Sprache wird verwendet? Auf welcher Plattform wird entwickelt? Welche Technologien werden eingesetzt.

%\section{Datenschutzrichtlinien}
%<Hier müssen die Datenschutzbestimmungen abgehandelt werden. Im normalfall wären IP Adressen personenbezogene daten und müssten geschützt werden. Es gibt da aber die außnahme in verbindung mit dem gesetz zur Vorratsdatenspeicherung. Darauf muss eingegangen werden um klar zu stellen unter welchen rechtlichen grundlagen die verarbeitung hier stattfinden kann. Dabei sollten auch die begriffe der anonymisierung und pseudonymisierung geklärt werden und was gemacht werden muss (wenn überhaupt).>